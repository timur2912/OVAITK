\documentclass[12pt]{article}
\usepackage[margin=1in]{geometry}
\usepackage[all]{xy}


\usepackage{amsmath,amsthm,amssymb,color,latexsym}
\usepackage{geometry}        
\geometry{letterpaper}    
\usepackage{graphicx}
\usepackage[russian]{babel}
\newtheorem{problem}{Задача}

\newenvironment{solution}[1][\it{Решение}]{\textbf{#1. } }{$\square$}
\newtheorem{theorem}{Теорема}[section]
\newtheorem{corollary}{Corollary}[theorem]
\newtheorem{lemma}[theorem]{Lemma}

\begin{document}
\noindent ОВАиТК 2024\hfill Домашнее задание 3 \\
Гаттаров Тимур Б05-304 (23/02/2024)

\hrulefill


\begin{problem}
Найдите число решений $x_{15}=e$ в группе $C_{54}$. Выпишите эти решения, если а - образующий группы.
\end{problem}
\begin{solution}
    Всего таких решений $\text{НОД}(15, 54) = 3$. Решением уравнения будут $a^k : 54|15k$.
    $$k_{min} = \frac{\text{НОК}(54, 15)}{15} = 18$$ 
    Решения уравнения: $x = \{a^0, a^{18}, a^{36}\}$
\end{solution}


\begin{problem}
Найти все первообразные корни в $\mathbb{Z}_{37}^*$
\end{problem}
\begin{solution}
    Подсчитаем количеством элементов в $\mathbb{Z}_{37}^*$:
    $$\varphi(37)=36$$
    Тогда по теореме Лагранжа возможные порядки $a$: $2, 4, 6, 9, 12, 18, 36$. Покажем, что 2 первообразный элемент. Достаточно показать, что $a^{12} \not \equiv 1 \bmod {37}$ и $a^{18} \not \equiv 1 \bmod {37}$.

$$ 2^{18}=32^3 \cdot 2^3 \equiv(-5)^3 \cdot 8 \equiv 25 \cdot(-40) \equiv(-12) \cdot (-3) \equiv 36 \bmod 37 $$ $$ & 2^{12}=32^2 \cdot 2^2 \equiv(-5)^2 \cdot 4 \equiv 25 \cdot 4 \equiv 26 \bmod 37$$

Все остальные первообразные корни представимы в виде $2^p$, где НОД $(p, 36)=1$. Их количество $\varphi(\varphi(37)) = (3^2 - 3) \cdot (2^2 - 2) = 12$.

Ответ:
$\{2^{1}, \text{ } 2^{5}, \text{ } 2^{7}, \text{ } 2^{11},\text{ }  2^{13}, \text{ }2^{17},\text{ } 2^{19}, \text{ }2^{23}, \text{ }2^{25}, \text{ }2^{29}, \text{ }2^{31}, \text{ }2^{35} \}$


\end{solution}

\begin{problem}
    Вычислить $12^{257} \bmod 17$
\end{problem}

\begin{solution}
        $$
    12^{257} = 12 \cdot 12^{256} = 12 \cdot 3^{256} \cdot 2^{128} \cdot 2^{128} =  12 \cdot 18^{128} \cdot 16^{32} \equiv  12 \cdot 1^{128} \cdot (-1)^{32} \bmod 17 = 12
    $$
\end{solution}


\begin{problem}
    Делится ли $25^{54}-1$ на $107 ?$
\end{problem}
\begin{solution}
    Воспользуемся малой теоремой Ферма:
    \begin{theorem}{Малая теорема Ферма}
    
        Пусть $a, p \in \mathbb{N}, \operatorname{GCD}(a, p)=1, p-$ простое. Тогда $a^{p-1} \equiv 1 \bmod p$.
    \end{theorem}
    $$
    25^{54}= 5^{108} = 25 \cdot 5^{107 - 1} \equiv 25 \cdot 1 \bmod{107} \equiv 25 \bmod{107}
    $$
    Из полученного очевидно, что $25^{54} - 1$ не делится на $107$.
\end{solution}
\begin{problem}
    Докажите, что в группе $S_8$ нет элементов порядка 56. Постройте элемент порядка 56 в какой-нибудь симметрической группе.
\end{problem}

\begin{solution}
    Воспользуемся утверждением: \textit{Порядок перестановки равен НОК длин всех циклов в её цикловом представлении.}

    Чтобы поличить НОК = $15$, необходимо наличие циклов длины 7 и 8 в $S_8$. Но $7 + 8 = 15 > 8$. Пример элемента порядка $56$ в группе $S_{15}$: $$(1,2,...7,8)(9,10,...14, 15).$$
\end{solution}

\begin{problem}
    Пусть $a=(154)(23), b=(12)(5364), a, b \in S_6$
    \begin{enumerate}
        \item Запишите перестановки в стандартном виде
        \item  Вычислите $a \circ b$
        \itemНайдите х, если $a \circ x=b$
        \item Запишите ответы в цикловом представлении
    \end{enumerate}
\end{problem}

\begin{solution}
    \begin{enumerate}
        \item $$a=\begin{pmatrix}
                1 & 2 & 3 & 4 & 5 & 6 \\
                5 & 3 & 2 & 1 & 4 & 6
                \end{pmatrix}
            $$
            $$ b =\begin{pmatrix}
                1 & 2 & 3 & 4 & 5 & 6 \\
                2 & 1 & 6 & 5 & 3 & 4
                \end{pmatrix}
            $$
        \item $$ c = a \circ b =\begin{pmatrix}
                1 & 2 & 3 & 4 & 5 & 6 \\
                3 & 5 & 6 & 4 & 2 & 1
                \end{pmatrix}
            $$

        \item $$ a^{-1} =\begin{pmatrix}
                5 & 3 & 2 & 1 & 4 & 6 \\
                1 & 2 & 3 & 4 & 5 & 6
                \end{pmatrix} = \begin{pmatrix}
                1 & 2 & 3 & 4 & 5 & 6 \\
                4 & 3 & 2 & 5 & 1 & 6
                \end{pmatrix}
            $$

            $$x = a^{-1} \circ b = \begin{pmatrix}
                1 & 2 & 3 & 4 & 5 & 6 \\
                3 & 4 & 6 & 1 & 2 & 5
                \end{pmatrix}$$

        \item $$a \circ b=(136)(25) \\ & $$
              $$x=(136524)$$
    \end{enumerate}
\end{solution}
\end{document}
б 