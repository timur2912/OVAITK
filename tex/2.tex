\documentclass[12pt]{article}
\usepackage[margin=1in]{geometry}
\usepackage[all]{xy}


\usepackage{amsmath,amsthm,amssymb,color,latexsym}
\usepackage{geometry}        
\geometry{letterpaper}    
\usepackage{graphicx}
\usepackage[russian]{babel}
\newtheorem{problem}{Задача}

\newenvironment{solution}[1][\it{Решение}]{\textbf{#1. } }{$\square$}


\begin{document}
\noindent ОВАиТК 2024\hfill Домашнее задание 2 \\
Гаттаров Тимур Б05-304 (15/02/2024)

\hrulefill


\begin{problem}
Пусть $G$ - группа нечетного порядка. Докажите, что каждый её элемент является квадратом некоторого другого.
\end{problem}

\begin{solution}
Так как порядок каждого элемента является делителем порядка группы, то все порядки нечётны. Пусть $e = a^{2p + 1}$. Тогда $a = ae = a^{(2p + 1) + 1} = a^{(p + 1)^2}$
\end{solution} 


\begin{problem}
Докажите, что элементы $a b c, b c a, c a b$ имеют одинаковый порядок (в случае, если каждый из них имеет конечный порядок).
\end{problem}

\begin{solution}
Пусть $(abc)^k = e$
$$
(abc)^k = abcabcabc...abcabc = e
$$


Домножим справа на $c^{-1}$:


$$
(abc)^k = abcabcabc...abcab = ec^{-1} = c^{-1}e
$$



Домножим слева на $c$:

$$
cabcabcabc...abcab = e = (cab)^k
$$

Аналогично доказывается для $bca$ до множением на $a^{-1}$ слева и на $a$ справа.
\end{solution}


\begin{problem}
 Пусть порядок $x$ равен 38. Найдите порядки $x^9$ и $x^6$.
\end{problem}

 \begin{solution} Порядок $m_1 = x^9$, $m_1 = \frac{38}{\text{НОД(38, 9)}} = 38$. Порядок $m_2 = x^6$, $m_2 = \frac{38}{\text{НОД(38, 6)}} = 19$.
 \end{solution}


\begin{problem}
    Пусть $C_n=<a>$ и НОД $(k, n) = 1$. Докажите, что $\exists b: b^k=a$.
\end{problem}

\begin{solution}
    Воспользуемся леммой доказанной на семинаре:
\begin{quotation}
Пусть $<a>=C_n$ и $b=a^k$. Доказать, что элемент $b$ тогда и только тогда будет образующим группы $\langle a\rangle$, когда числа $n$ и $k$ взаимно просты.
\end{quotation}


Тогда имеем $a^k$ - образующий жлемент. Из определения образующиего элемента получаем: $\exists t : (a^k)^p = a$. Продолжим равенство и явно предъявим $b$:
$$
(a^k)^p = a = a^{kp} = (a^p)^k
$$

Получили $\exists p: b^k = (a^p)^k = a$.

\end{solution}


 \begin{problem}
 Докажите, что любая группа простого порядка циклическая.
 \end{problem}

\begin{solution}
    Так как порядок элемента является делителем порядка группы, то в нашем случае порядок элемента либо 1, либо какое-то простое число $p$. Тогда если порядок элемента $ord(a) = 1$, то $a = e$, иначе если $ord(a) = p$ (все остальные элементы) $\forall b \exists k \in Z : a^k = b, k = p$ и тогда группа циклическая по определению. 
\end{solution}
 \begin{problem}
     Найдите количество элементов в группе $\mathbb{Z}_{108}^*$.
 \end{problem}

\begin{solution}
    Посчитаем значение функции Эйлера $\varphi(108)$:
    $$
    \varphi(108) = \varphi(2^2) \cdot \varphi(3^3) = (2^2 - 2) \cdot (3^3 - 3^2) = 36
    $$
\end{solution}
\begin{problem}
    Пусть $d \mid n$. Докажите, что в $C_n$ ровно $\varphi(d)$ элементов порядка $d$. Используя этот факт докажите, что $\sum_{d \mid n} \varphi(d)=n$
\end{problem}

\begin{solution}
    Рассмотрим элемент $a^p$. Аналогично Задаче 3 $d = ord(a^p) = \frac{n}{\text{НОД($n, p$)}}$. Перепишем в удобном в виде:
    $$
    \text{НОД($n, p$)}=\frac{n}{d}=k
    $$
    $$
    \text{НОД($d, \frac{pd}{n}$)}=  \text{НОД($d, \frac{p}{k}$)} = 1
    $$

    Тогда всего чисел $\frac{p}{k} < d$ и взаимно простых с $d$ $\varphi(d)$.

    Воспользуемся тем что порядок каждого элемента группы является делителем числа $d$. Тогда $\sum_{d \mid n} \varphi(d)$ является суммой колечества элементов по всем порядкам, которая равна количеству элементов группе $n$.
\end{solution}
\end{document}
б 