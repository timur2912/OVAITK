\documentclass[12pt]{article}
\usepackage[margin=1in]{geometry}
\usepackage[all]{xy}


\usepackage{amsmath,amsthm,amssymb,color,latexsym}
\usepackage{geometry}        
\geometry{letterpaper}    
\usepackage{graphicx}
\usepackage[russian]{babel}
\newtheorem{problem}{Задача}

\newenvironment{solution}[1][\it{Решение}]{\textbf{#1. } }{$\square$}


\begin{document}
\noindent ОВАиТК 2024\hfill Домашнее задание 1 \\
Гаттаров Тимур Б05-304 (10/02/2024)

\hrulefill


\begin{problem}
Образуют ли группу:


(a) положительные рациональные числа относительно деления.


(b) матрицы $3 x 3$ с целыми коэффициентами и детерминантом 1 относительно умножения матриц.
\end{problem}

\begin{solution}
    

(a) Нет. Так как для группы на множестве рациональных относительно операции деления не выполнена ассоциативность:

 $$
\frac{\frac{a}{b}}{c} = \frac{a}{bc} \neq \frac{a}{\frac{b}{c}} = \frac{ac}{b}
 $$


(b) Да. Замкнутость доказывается очевидно: при произвдении двух матриц (3x3) получается матрица (3x3). Заметим что, коэффициенты полученной матрицы целые числа, так происходит только сложение и умножение целых чисел. Ассоциативность произвдения матриц (если оно определено) доказывается в курсе линейной алгебы. Существует нейтральный элемент: $$E = 
\[
\begin{bmatrix}
1 & 0 & 0 \\
0 & 1 & 0 \\
0 & 0 & 1 \\
\end{bmatrix}
\]
$$

Несложно доказывается существование $A^{-1}$:  $A^{-1}A = E$, причём $A^{-1}$ должен иметь целые коэффициенты. Для элементов обратной матрицы существует формула: $A^{-1}=\frac{1}{\operatorname{det} A} \cdot A^* = A^* $ (в нашем случае)
где $A^*$ - матрица алгебраических дополнений, все коэффициенты которой целые числа, так как они являются детерминантами миноров матрицы с целыми коэффиентами (проще говоря при сложении, вычитании, умножении целых числе - результат целое число). 

Единственность $E$ и $A^{-1}$ доказывается в курсе линейной алгебры.
\end{solution} 


\begin{problem}
Говорят, что элементы $a$ и $b$ группы $G$ коммутируют, если выполнено $a b=b a$. Докажите, что если $a$ и $b$ коммутируют, то $a$ и $b^{-1}$ коммутируют.
\end{problem}

\begin{solution}
$$ab = ba$$
$$abb^{-1} = bab^{-1} = a$$
$$bab^{-1} = ea = bb^{-1}a$$
$$ab^{-1} = b^{-1}a$$
\end{solution}

\newpage

\begin{problem}
    Для любого элемента $a$ группы $G$ выполнено, что $a^2=e$. Докажите, что группа абелева.
\end{problem}

 \begin{solution}
Запишем условие для $a$ и $b$:
$$a^{2} = aa = e$$
$$b^{2} = bb = e$$
Тогда $$a^2b^2 = ee = e$$
Запишем условие для $ab$:
$$(ab)^{2} = abab = ee = e $$
Тогда:
$$abab = e = aabb$$
$$aba = aabb$$
$$ba = ab$$
Тогда группа абелева.
 \end{solution}

 \begin{problem}
     Пусть $H$ - подмножество группы $G$. Докажите, что $H$ - подгруппа тогда и только тогда, когда $H$ непусто и $\forall x, y \in H: x y^{-1} \in H$.
 \end{problem}
 
\begin{solution}

$\Rightarrow$ Из определения непустой подгруппы, очевидно доказывается свойство $\forall x, y \in H: x y^{-1} \in H$: $H$ - подгруппа: $\forall y \text{ } \exists \text{ }y^{-1} \in H, x \in H \rightarrow x y^{-1} \in H$

$\Leftarrow$ Покажем из того, что $H$ непусто и $\forall x, y \in H: x y^{-1} \in H$, то $H$ - группа.

Покажем существование нейтрального элемента: пусть $x = y$, тогда
$\forall x \in H \hookrightarrow x x^{-1} = e \in H$. 

Покажем существование обратного элемента: пусть $x = e$, тогда
$\forall y \in H \hookrightarrow e y^{-1} = y^{-1} \in H$. 

Ассоциативность следует из того что $H$ - подмножество группы.


Покажем замкнутость:
$$\forany x, y \in H \hookrightarrow xy^{-1} \in H$$
$$ x(xy^{-1})^{-1} = xyx^{-1} \in H$$
$$ xyx^{-1} (x^{-1})^{-1} = xy \in H $$
\end{solution}

\begin{problem}
Найдите все конечные подгруппы группы целых чисел по сложению
\end{problem}

\begin{solution}
Воспользуемся тем, что подгруппа $H$ конечна, тогда выберем:
$$M = \max(H) $$
$$m = \min(H) $$
Тогда требуется, чтобы были выполнены следующие неравенства:
$$m + m \geq m \rightarrow m \geq 0$$
$$M + M \leq M \rightarrow M \leq 0$$
Тогда группа $G = (G, +), \text{ } G = \{0\}$
\end{solution}
\end{document}
