\documentclass[12pt]{article}
\usepackage[margin=1in]{geometry}
\usepackage[all]{xy}


\usepackage{amsmath,amsthm,amssymb,color,latexsym}
\usepackage{geometry}        
\geometry{letterpaper}    
\usepackage{graphicx}
\usepackage[russian]{babel}
\newtheorem{problem}{Задача}

\newenvironment{solution}[1][\it{Решение}]{\textbf{#1. } }{$\square$}
\newtheorem{theorem}{Теорема}[section]
\newtheorem{corollary}{Corollary}[theorem]
\newtheorem{lemma}[theorem]{Lemma}

\begin{document}
\noindent ОВАиТК 2024\hfill Домашнее задание 5 \\
Гаттаров Тимур Б05-304 (08/03/2024)

\hrulefill

\begin{problem}
    Укажите классы сопряжённости в $S_5$.
\end{problem}

\begin{solution}
    В общем случае число классов сопряжённости в симметрической группе $S_n$ равно количеству разбиений числа $n$, так как каждый класс сопряжённости соответствует в точности одному разбиению перестановки $\{1,2, \ldots, n\}$ на циклы.
    Число разбиений для $l(5) = 7$ : \{$\{5\}, \{4, 1\}, \{3, 2\}, \{3, 1, 1\}, \{2, 2, 1\}, \{2, 1, 1, 1\}, \{1, 1, 1, 1, 1\}\}$
\end{solution}
\begin{problem}
    Укажите нормальные подгруппы в $S_3$.
\end{problem}

\begin{solution}
    Перечислим все перестановки $S_3$:
    $$
    S_3 = \{e, (12), {23}, (31), (123), (132)\}
    $$

    Очевидно что, $e$ и $S_3$ являются нормальными подгруппами. Проверим остальные: $H=\{e,(12)\}$ является группой, но не является нормальной (для $x=(23))$. Аналогичные рассуждения приводятся и для $H=\{e,(13)\}, \text{ } H = \{e,(23)\}$.
    
    $H=\{e,(123),(132)\}$ - подгруппа, при том индекса 2 , а подгруппы иднекса 2 всегда нормальные.


    Ответ: $\{e\}$, $\{e,(123),(132)\}$, $S_3$
    
\end{solution}

\begin{problem}
    Пусть $G$ - группа, $H$ - её нормальная подгруппа, докажите, что $\forall x \in H:[x] \in H$. (То есть что классы сопряженности должны целиком лежать в нормальной подгруппе)
\end{problem}

\begin{solution}
    Подгруппа $H$ нормальна тогда и только тогда, когда для любых $h \in$ $H$ и $g \in G$ выполнено $g h g^{-1} \in H$. То есть для любого $h \in H$ выполнено $C(h) \subset H$.
\end{solution}

\begin{problem}
    Используя прошлую задачу докажите, что не существует нетривиальных (не единичный элемент и не сама группа) нормальных подгрупп группы $A_5$.
\end{problem}


\begin{solution}
    Группа $A_5$ представима из циклов вида $(ab)(cd)(e)$  или $(abc)(d)(e)$ - остальные тривиальными.
    Воспользуемся утвержением, доказанным в \textit{Задаче 3}. Если существует нетривиальная нормальная подгруппа, то она должна либо целиком содержать классы смежности вида $(ab)(cd)(e)$  или $(abc)(d)(e)$.


    Приведём контрпример, где класс смежности $(ab)(cd)(e)$ не образует подгруппу: $(12)(34) \circ (23)(45)=(24531)$. Аналогичный контрпример и для $(abc)(d)(e)$ : $(123) \circ (234)=(21)(34)-$ не группа.
\end{solution}
\begin{problem}
    Докажите, что нормальная подгруппа индекса $k$ содержит все элементы, порядки которых взаимно просты с $k$.
\end{problem}

\begin{solution}
    Пусть $H$ - нормальная подгруппа $G$ с индексом $k$. Пусть  также есть $x$ порядка $n$. НОД$(n, k)=1$ :
    $$
    x^n=e \in H
    $$
    
    Фактор-группа $G / H$ имеет порядок $k$, поэтому $(x H)^k=H$. Воспользуемся нормальностью подгруппы:
    $$
    (x H)^k=x^kH=H \Longrightarrow x^k \in H
    $$
    Так как НОД$(n, k)=1$, существуют такие целые числа $x, y$, что
    $$
    k x+n y=1
    $$
    
    Возведем в эту степень $x$ :
    $$
    x=x^1=\left(x^k\right)^x \cdot\left(x^n\right)^y \in H
    $$
    
    Утверждение доказано.
\end{solution}
\end{document}
