\documentclass[12pt]{article}
\usepackage[margin=1in]{geometry}
\usepackage[all]{xy}


\usepackage{amsmath,amsthm,amssymb,color,latexsym}
\usepackage{geometry}        
\geometry{letterpaper}    
\usepackage{graphicx}
\usepackage[russian]{babel}
\newtheorem{problem}{Задача}

\newenvironment{solution}[1][\it{Решение}]{\textbf{#1. } }{$\square$}
\newtheorem{theorem}{Теорема}[section]
\newtheorem{corollary}{Corollary}[theorem]
\newtheorem{lemma}[theorem]{Lemma}

\begin{document}
\noindent ОВАиТК 2024\hfill Домашнее задание 4 \\
Гаттаров Тимур Б05-304 (27/02/2024)

\hrulefill


\begin{problem}
Найдите порядок перестановки (123)(4567) в $S_8$. Найдите количество сопряженных к ней. (Сопряженной к перестановке $a$ называются перестановки, представимые в виде $\sigma a \sigma^{-1}$ )
\end{problem}

\begin{solution}
Воспользуемся утвержднеием: \textit{порядок перестановки равен НОК длин всех циклов в её цикловом представлении.} Порядок перестановки $k = \text{НОК}(1, 3, 4) = 12$.

Воспользуемся ещё одним утвержением: \textit{для любой перестановки $\tau$ сопряжение ее произвольной перестановкой на любом количестве символов формирует перестановку, которая имеет точно такую же структуру разложения в циклы, как и $\tau$.} 

Тогда выберем в первый цикл 4 элемента из 8, во второй 3 элемента из оставшихся 5 и рассмотрим всевозможные перестановки в этих циклах, кроме циклических перестановок: зафиксируем первое в них число и перемешаем все остальные.
$$
C_8^4\cdot C_4^3 \cdot 3! \cdot 2!
$$
\end{solution}
\begin{problem}
Найдите все решения уравнения $\sigma^2=(123)$ в $S_6$.
\end{problem}

\begin{solution}
    Воспользуемся утверждением: \textit{при возведении цикла длины $n$ в степень $k$ цикла распадается на цикл длины $\frac{n}{\text{НОД($n, k$)}}$}.

    При возведении цикла длины 6 он распадается на 2 цикла длины 3, длины 5 - в цикла длины 5, длины 4 - в в цикл 2 цикла длины 2, цикл длины 2 - в 2 цикла длины 1. Таким образом нас интересуют только циклы длины 3. Тогда рассмотрим 4 случая:
    $$
    (abc)^2 = (bac) = (123)
    $$
    
    $$
    (abc)^2(cd)^2(e)^2 = (bac) = (123)(4)(5)(6)
    $$

    $$
    (abc)^2(c)^2(de)^2 = (bac) = (123)(4)(5)(6)
    $$

    $$
    (abc)^2(ce)^2(d)^2= (bac) = (123)(4)(5)(6)
    $$
    Ответ: $\sigma = (213)$, $(213)(45)$, $(213)(56)$, $(213)(46)$
\end{solution}

\begin{problem}
Найдите $\left(\begin{array}{llllllll}1 & 2 & 3 & 4 & 5 & 6 & 7 & 8 \\ 2 & 7 & 5 & 1 & 6 & 3 & 4 & 8\end{array}\right)^{2030}$
\end{problem}

\begin{solution}
$((1274)(356))^{2030}=(1274)^2(356)^2=(17)(24)(365)$.
\end{solution}

\newpage
\begin{problem}
Найдите наименьший $n$ такой, что группа $C_{12}$ (циклическая группа порядка 12) изоморфна одной из подгрупп $S_n$.
\end{problem}

\begin{solution}
    \textit{(Аналогично разобранному в учебнике "<Основы высшей алгебры и теории кодирования"> Ю. И. Журавлёв, Ю. А. Флёров, М. Н. Вялый примеру 6.28)}

    
    Теорема Кэли гарантирует, что $C_n \cong G<S_n$. Из её доказательства ясно даже, что это за подгруппа $G$ : она порождена циклом длины $n$.

    Однако это не наименьшее $k$ для которого в $S_k$ существует подгруппа, изоморфная $C_n$. Действительно, любой элемент группы, порядок которого равен $n$, порождает подгруппу, изоморфную $C_n$.
    
    Вспоминая формулу для порядка перестановки, получаем такую неявную характеризацию тех $k$, для которых в $S_k$ есть элемент порядка $n$ : существует такое разбиение $k$ на слагаемые, $k=k_1+k_2+\cdots+k_t$, что $\operatorname{HOK}\left(k_1, k_2, \ldots, k_t\right)=n$. 
    $$k = 3 + 4 = 7\text{, НОК}(3, 4) = 12$$
\end{solution}

\begin{problem}
Укажите все смежные классы группы $C_{12}$ по подгруппе порядка 3.
\end{problem}
\begin{solution}
Так как $H$ подгруппа, то она сожержит в себе $e$. Пусть подгруппа содержит в себе элемент $x$, но тогда она и содержит элемент $x^{-1} : xx^{-1} = e$. Но длина подгруппы по условию $3$. Тогда имеем группу:
$$
H = \{e, x, x^{-1}\}
$$

Порождающим элементом в $C_{12}$ является $1$. Найдём подгруппу $H$. Воспользуемся теоремой Лагранжа и условием количества элементов в группе:
$$\frac{12}{\text{НОД}(12, k)}=3 \longrightarrow k \in\{4,8\}$$
где $x = 1^k$. Получили интересующую нас подгруппу $H = {0, 4, 8}$. Перечислим все классы смежности с учётом коммутативности операции $+$ : $\{0, 4, 8\}; \{1, 5, 9\}; \{2, 6, 10\}; \{3, 7, 11\}$
\end{solution}
\end{document}
б 