\documentclass[12pt]{article}
\usepackage[margin=1in]{geometry}
\usepackage[all]{xy}
\usepackage{hyperref}
\hypersetup{
    colorlinks=true,
    linkcolor=blue,
    filecolor=magenta,      
    urlcolor=cyan,
    pdftitle={Overleaf Example},
    pdfpagemode=FullScreen,
    }

\urlstyle{same}
\usepackage{epigraph}


\usepackage{amsmath,amsthm,amssymb,color,latexsym}
\usepackage{geometry}        
\geometry{letterpaper}    
\usepackage{graphicx}
\usepackage[russian]{babel}
\newtheorem{problem}{Задача}

\newenvironment{solution}[1][\it{Реше\textit{}ние}]{\textbf{#1. } }{$\square$}
\newtheorem{theorem}{Теорема}[section]
\newtheorem{corollary}{Corollary}[theorem]
\newtheorem{lemma}[theorem]{Lemma}
\DeclareMathOperator{\Ima}{Im}
\DeclareMathOperator{\Kera}{Ker}

\begin{document}
\noindent ОВАиТК 2024\hfill Домашнее задание 9 \\
Гаттаров Тимур Б05-304 (14/04/2024)

\hrulefill


\epigraph{Всё это — как полотна Сальвадора Дали:

Ищешь ответы снаружи, но ответы внутри.}{"Обернись"


Баста}

\begin{problem}
 Даны многочлены $6 x^3+2 x^2+x+2$ и $x^2+4 x+3$. Найдите их НОД и коэффициенты, с которыми данные нужно сложить данные многочлены для его получения.

 
(a) над полем вычетов по модулю 5


(b) над полем вычетов по модулю 7
\end{problem}

\begin{solution}
\href{https://imgbly.com/ib/cu9DA7tHFE}{Решение a}
\href{https://imgbly.com/ib/NrkHrM7Bxy}{Решение b}
\end{solution}


\begin{problem}
     Являются ли полями $\mathbb{F}_5[x] /\left(x^2+1\right), \mathbb{F}_7[x] /\left(x^2+1\right)$ ? Если нет, то найдите делители нуля в соответствующем кольце.
\end{problem}

\begin{solution}
    \href{https://imgbly.com/ib/uXPWMuOqMh}{Решение}
\end{solution}

\begin{problem}
     Является ли полем кольцо вычетов $\mathbb{R} /\left(x^2+1\right)$ ? Докажите, что оно изоморфно $\mathbb{C}$. (Гомоморфизм колец - сохранение обоих операций).
\end{problem}


\begin{solution}
    \href{https://imgbly.com/ib/bGJd5D8noI}{Решение}
\end{solution}


\begin{problem}
     Найдите нильпотентные элементы в $\mathbb{Z}_7[x] /\left(x^2+x-5\right)$
\end{problem}


    
\begin{solution}
    \href{https://imgbly.com/ib/IFFKiaT9OH}{Решение}
\end{solution}
\end{document}
